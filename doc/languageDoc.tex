\documentclass[11pt,a4paper]{article}
\usepackage[utf8]{inputenc}
\usepackage[english]{babel}
\usepackage[left=2cm,right=2cm,top=2cm,bottom=2cm]{geometry}

\usepackage{framed}

\author{Tiago Lobato Gimenes}
\title{X-CubeSat Controller language documentation}

\begin{document}
\maketitle

\section*{Introduction}
In this file you will be able to find the definitions of the language used in the X-CubeSat Controller.

\begin{center}
\line(1,0){500}
\end{center}

\section*{Types:}
\begin{itemize}
\item \textbf{string}
\item \textbf{data}
\item \textbf{bool}
\item \textbf{int}
\item \textbf{void}
\end{itemize}

\section*{Functions:}
\begin{itemize}
\item \textbf{bool SEND( string )}
\item \textbf{bool SAVE( data, string )}
\item \textbf{bool IF( bool )}
\item \textbf{bool EQ( int, int )}
\item \textbf{bool LL( int, int )}
\item \textbf{bool LEQ( int, int )}
\item \textbf{string FILE( string )}
\item \textbf{string APPENDDATE( string )}
\item \textbf{string FORMAT( string )}
\item \textbf{data RECEIVE( string )}
\item \textbf{void DECLARE( type, string)}
\item \textbf{void SET(string, type)}
\end{itemize}

\begin{center}
\line(1,0){500}
\end{center}

\section*{SEND()}
\begin{framed}
\subsection*{bool SEND( string )}
\begin{itemize}
\item \textbf{Params:} string to be sent to the interface (often the modem).
\item \textbf{Return:} boolean value indicating if the send process was successfully accomplished (true) or not (false).
\end{itemize}
\end{framed}

\section*{RECEIVE()}
\begin{framed}
\subsection*{data RECEIVE( string )}
\begin{itemize}
\item \textbf{Params:} string with the name of the format that will be received.
\item \textbf{Return:} read data from the interface  (modem) with the right format associated.
\end{itemize}
\end{framed}

\section*{SAVE()}
\begin{framed}
\subsection*{bool SAVE( data, string )}
\begin{itemize}
\item \textbf{Params:}
\begin{enumerate}
\item data with the data to be saved.
\item string with the file name that the data will be saved.
\end{enumerate}
\item \textbf{Return:} boolean indicating if the save was successful (true) or not (false).
\end{itemize}
\end{framed}

\section*{IF(){}}
\begin{framed}
\subsection*{bool IF (bool)}
\begin{itemize}
\item \textbf{Params:} if true to run what is inside of {}, else to continue.
\item \textbf{Return:} true if it should run the {} or false if it should continue.
\end{itemize}
\end{framed}

\section*{DECLARE()}
\begin{framed}
\subsection*{void DECLARE( type, string )}
\begin{itemize}
\item \textbf{Params:}
\begin{enumerate}
\item the type of the new variable.
\item the name of the new variable.
\end{enumerate}
\item \textbf{Return:} void
\end{itemize}
\end{framed}

\section*{EQ()}
\begin{framed}
\subsection*{bool EQ(int, int)}
\begin{itemize}
\item \textbf{Params:}
\begin{enumerate}
\item first element to compare.
\item second element to compare.
\end{enumerate}
\item \textbf{Return:} true if first element equals the second one, false otherwise.
\end{itemize}
\end{framed}

\section*{LL()}
\begin{framed}
\subsection*{bool LL(int, int)}
\begin{itemize}
\item \textbf{Params:}
\begin{enumerate}
\item first element to compare.
\item second element to compare.
\end{enumerate}
\item \textbf{Return:} true if first element is less than the second one, false otherwise.
\end{itemize}
\end{framed}

\section*{LEQ()}
\begin{framed}
\subsection*{bool LEQ(int, int)}
\begin{itemize}
\item \textbf{Params:}
\begin{enumerate}
\item first element to compare.
\item second element to compare.
\end{enumerate}
\item \textbf{Return:} true if first element is less or equal the second one, false otherwise.
\end{itemize}
\end{framed}

\section*{FILE()}
\begin{framed}
\subsection*{string FILE( string )}
\begin{itemize}
\item \textbf{Params:} string with the name of the file to be opened.
\item \textbf{Return:} string with the content of the file.
\end{itemize}
\end{framed}

\section*{APPENDDATE()}
\begin{framed}
\subsection*{string APPENDDATE( string )}
\begin{itemize}
\item \textbf{Params:} string.
\item \textbf{Return:} the parameter appended with the date (dd/mm/yyyy:hh:mm:ss) of execution of this function.
\end{itemize}
\end{framed}

\section*{FORMAT()}
\begin{framed}
\subsection*{string FORMAT( string )}
\begin{itemize}
\item \textbf{Params:} string with the name of the format.
\item \textbf{Return:} string with the name of the format.
\end{itemize}
\end{framed}

\section*{SET()}
\begin{framed}
\subsection*{void SET(string, type)}
\begin{itemize}
\item \textbf{Params:}
\begin{enumerate}
\item variable name
\item type with the correct data to set the variable
\end{enumerate}
\item \textbf{Return:} void
\end{itemize}
\end{framed}

\end{document}